\section{Gradient Descent on Manifolds}

\begin{frame}{First-order Optimality Conditions}
  We need the notion of critical points such that the Fermat theorem is reserved. Again, curves help: a point $x\in \M$ is a critical point of $f:\M\to \RR$ if the velocity of any curve passing through $x$ is $0$.

  \begin{mydefinition}[Critical point]
  A point $x\in \M$ is a critical point of $f:\M\to \RR$ if for any smooth curve $\gamma:(-\epsilon,\epsilon) \to \M$ with $\gamma(0) = x$, we have
  $$(f\circ \gamma)'(0) \ge 0.$$
  \end{mydefinition}

  The definition uses $(f\circ \gamma)'(0) \ge 0$ and that is equivalent: we can consider $t\mapsto c(t)$ and $t\mapsto c(-t)$.

  \begin{myproposition}
    A point $x\in \M$ is a critical point of $f:\M\to \RR$ if and only if $\grad f(x) = 0$.
  \end{myproposition}

\end{frame}

\begin{frame}{Riemannian Gradient Descent}
  The framework is the iteration
  $$x^{(k+1)} = R_{x^{(k)}}\left(-\alpha^{(k)}\grad f(x^{(k)})\right),$$
  where $x^{(0)}$ is initialized in $\M$ and $\alpha^{(k)} > 0$ is the step size.

  \begin{mytheorem}
    Let $f:\M\to \RR$ be a smooth, bounded-below function on a Riemannian manifold $\M$. Choose a retraction $R$ such that $\grad f$ is $L$-Lipschitz. If $\alpha^{(k)} \in [0, 2/L]$ for every step, then the sequence $\{x^{(k)}\}$ generated by Riemannian gradient descent satisfies
    $$\lim_{k\to\infty}\|\grad f(x^{(k)})\| = 0.$$
  \end{mytheorem}
\end{frame}

% \begin{frame}{Convergence Results}

% \end{frame}