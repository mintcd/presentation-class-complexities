\section{Manifolds}

\subsection{Embedded Submanifolds of a Linear Space}

\begin{frame}{Motivation}
  A subset $\M$ of $\RR^d$ is an \textcolor{gustave}{embedded manifold of dimension $n$}  if for each point $x\in\M$, there exists a neighborhood in $\M$ of $x$ (i.e. $\M\cap U$ for some open set $U\subset \RR^d$ containing $x$) that is \textcolor{gray}{approximate} to an open subset of $\RR^n$.

  \onslide<2->{
    \begin{figure}
      \centering
      % https://www.geogebra.org/classic/r7dwyts9
      \begin{minipage}{0.4\textwidth}
        \centering
        \includegraphics[width=0.7\textwidth]{img/sphere.png}
        \caption*{A sphere is a manifold}
      \end{minipage}
      \hspace{0.5cm}
      % https://www.geogebra.org/classic/gjmghdym
      \begin{minipage}{0.4\textwidth}
        \centering
        \includegraphics[width=0.7\textwidth]{img/cone.png}
        \caption*{A cone is not a manifold because every neighborhood of the tip (in red) cannot be approximated by a plane}
      \end{minipage}
    \end{figure}
  }

  % \textcolor{red}{
  %   Intuition for why $\RR^{d-n}$: because we can think of the manifold as being defined by d-n constraints in $\RR^d$.
  % }
\end{frame}

\begin{frame}{Embedded Submanifolds of a Linear Space}
  We consider \textcolor{gustave}{smooth} spaces: for each $x\in\M$, there exists an open set $U\subset \RR^d$ containing $x$ and a smooth map $h: U \to \RR^{d-n}$ such that $M\cap U = h^{-1}(\{0\})$.

  By being approximate to $\RR^n$, we mean that for any direction $v\in\RR^d$ that is a \textcolor{gray}{tangent vector} to $\M$ at $x$, we have
  $$h(x+tv) = o(t).$$

  We rely on \textcolor{gustave}{curves} for the definition of tangent vectors (and also later definitions).

  \begin{mydefinition}[Tangent space]
  The tangent space $T_x\M$ at a point $x\in\M$ is the set of all tangent vectors to $\M$ at $x$ i.e.
  $$T_x\M = \{\gamma'(0) \,|\, \gamma:(-\epsilon,\epsilon) \to \M \in \C^\infty(-\epsilon,\epsilon), \gamma(0) = x \}.$$
  \end{mydefinition}


\end{frame}

\begin{frame}{Embedded Submanifolds of a Linear Space}
  We can use Taylor expansion to write
  $$h(x+tv) = h(x) + t \mathrm{D}h(x)[v] + o(t) = t \mathrm{D}h(x)[v] + o(t).$$

  \begin{myproposition}
    For every $x\in \M$, we have $$T_x\M \subseteq \text{ker}(\mathrm{D}h(x)).$$
  \end{myproposition}

  \begin{proof}
    Let $v\in T_x\M$, then there is a smooth $\gamma$ such that $\gamma(0) = x$ and $\gamma'(0) = v$. Consider $g(t) = h(\gamma(t))$. Since $\gamma(t)\in \M$ for all $t$, we have $g(t) = 0$ for all $t$. Thus, $g'(0) = 0$. By chain rule,
    $$g'(0) = \mathrm{D}h(\gamma(0))[\gamma'(0)] = \mathrm{D}h(x)[v] = 0.$$
    Hence, $v\in \text{ker}(\mathrm{D}h(x))$.
  \end{proof}

\end{frame}

\begin{frame}{Embedded Submanifolds of a Linear Space}
  Consider two possible issues:
  \begin{enumerate}
    \item If $T_x\M \subsetneq \text{ker}(\mathrm{D}h(x))$. Since then there are vectors in $\text{ker}(\mathrm{D}h(x))$ that are not tangent to $\M$ at $x$ but can be used to approximate $\M$ near $x$.
          \begin{itemize}
            \item  For example, define cone shown previously by $h(x,y,z) = z^2 - x^2 - y^2$. At the tip $(0,0,0)$, we have $\mathrm{D}h(0,0,0) = [0 \; 0 \; 0]$ and $\ker \mathrm{D}h(0,0,0) = \RR^3$. So any vector in $\RR^3$ can be used to approximate the cone near the tip. But $v = (0,0,1)$ is not a tangent vector.
          \end{itemize}
          Hence, we want $T_x\M = \text{ker}(\mathrm{D}h(x))$.
    \item Now $T_x\M$ is a linear space. So we also want $\dim T_x\M$ to be constant for all $x\in \M$.
  \end{enumerate}

  Suppose that $\dim T_x\M = n$ for all $x\in \M$. Then, from the previous proposition, we have
  $$n = \dim T_x\M = \dim \text{ker}(\mathrm{D}h(x)) = d - \rank \mathrm{D}h(x).$$
  Or $\rank \mathrm{D}h(x) = d - n$.

  Conversely, we can show that if $\rank \mathrm{D}h(x) = d - n$, then the two conditions above are satisfied.
\end{frame}

\begin{frame}{Embedded Submanifolds of a Linear Space}
  \begin{mydefinition}
  A subset $\M$ of $\RR^d$ is an embedded submanifold of dimension $n$ if for each $x\in\M$, there exists an open set $U\subset \RR^d$ containing $x$ and a smooth map $h: U \to \RR^{d-n}$ such that $$M\cap U = h^{-1}(\{0\}) \text{ and } \forall x\in \M\cap U, \rank\,\mathrm{D}h(x) = d-n.$$
  \end{mydefinition}

  \begin{myproposition}
    Using the convention that $\RR^0 = \{0\}$, every open subset of $\RR^d$ is a $d$-dimensional embedded submanifold of $\RR^d$.
  \end{myproposition}

  \begin{myproposition}
    Let $\M\subset\RR^m$ and $\N\subset\RR^n$ be embedded submanifolds of dimensions $k$ and $\ell$ respectively. Then, $\M\times \N \subset \RR^{m+n}$ is an embedded submanifold of dimension $k+\ell$.
  \end{myproposition}

  % \textcolor{red}{We may add the theorem that this definition is equivalent to the diffeomorphism definition.}
\end{frame}

\subsection{Examples of Optimization on Manifolds}

\begin{frame}{Eigenvalues and Singular Values}
  The problem of finding the largest eigenvalue of a symmetric matrix $A\in \RR^{d\times d}$ is stated as
  $$\max_{x\in\RR^d} \langle x, A x \rangle \quad \text{s.t. } \|x\|_2 = 1.$$

  The constraint $\|x\|_2 = 1$ defines the unit sphere $S^{d-1} = \{x\in\RR^d \,|\, \|x\|_2^2 - 1 = 0\}$.

  The function $h(x) = \|x\|_2^2 - 1$ is smooth and $\mathrm{D}h(x) = 2x^\top$, which has rank 1 for all $x\in S^{d-1}$. Hence, $S^{d-1}$ is an embedded submanifold of $\RR^d$ of dimension $d-1$.

  \pause

  Similarly, the problem of finding the largest singular value of $M\in \RR^{m\times n}$ can be written as
  $$\max_{x\in S^{m-1}, y\in S^{n-1}} \langle x, M y \rangle,$$
  which is an optimization problem on the manifold $S^{m-1} \times S^{n-1}$.
\end{frame}

\begin{frame}{Principal Component Analysis}
  Let $X\in\RR^{d\times n}$ be the data matrix with $n$ samples of dimension $d$. The goal is to find an orthogonal basis of dimension $r$ that maximizes the weighted variance of the data when projected onto the basis.
  $$\max_{U\in \RR^{d\times r}} \text{tr}(U^\top X W X^\top U) \quad \text{s.t. } U^\top U = I_r.$$

  The set $\text{St}(r,d) = \{U\in \RR^{d\times r} \,|\, U^\top U = I_r\}$ is called the Stiefel manifold.
\end{frame}